\documentclass[11pt,a4paper]{article}
\usepackage{xcolor}
\usepackage{CJKutf8}
\usepackage{graphicx}
\usepackage{amsmath}
\usepackage{braket}
\usepackage{caption}
\usepackage{bm}
\usepackage{geometry}
\geometry{a4paper,scale=0.8}

\captionsetup{font={scriptsize}}

\begin{CJK}{UTF8}{gbsn}
\begin{document}

\title{Notes on PRB.67.092408}
% \author{Li Gaoyang}
\date{}
% \date{\today}
\maketitle
% \tableofcontents

% \newpage
\section{spin field effect transistor}
A type of quantum field effect transistor that operates purely on the flow of spin current in the absence of charge current. The rotating field induces a time-independent dc spin current, and at the same time generates no charge cur- rent. The physical principle of our SFET is due to a spin flip mecha- nism provided by the field.
\section{Hamiltonian}
\begin{equation}
\begin{split}
H=&\sum_{k, \sigma, \alpha=L, R} \epsilon_{k} C_{k \alpha \sigma}^{+} C_{k \alpha \sigma}+\sum_{\sigma}\left[\epsilon+\sigma B_{0} \cos \theta\right] d_{\sigma}^{+} d_{\sigma}\\
&+H^{\prime}(t)+\sum_{k, \sigma, \alpha=L, R}\left[T_{k \alpha} C_{k \alpha \sigma}^{+} d_{\sigma}+\mathrm{c.c.}\right]
\end{split}
\end{equation}
We assume that there is only on orbit in the scattering region.
\[
\epsilon_{Lk} = \epsilon_{Rk} = \epsilon_{k}.
\]
A rotating magnetic field is
\begin{equation}
\textbf{B} = B_{0}sin\theta [cos(\omega t) \hat{x} + sin(\omega t) \hat{y}] + B_{0}cos\theta \hat{z}.
\end{equation}
A counterclock- wise rotating field allows a spin-down electron to absorb a photon and flip to spin-up, and it does not allow a spin-up electron to absorb a photon and flip to spin-down.

The scattering region is characterized by an energy level$\epsilon = \epsilon_{0} - qV_{g}$, controlled by the gate voltage $V_{g}$.

We solve the transport properties (charge and spin currents) of the model in both adiabatic and nonadiabatic regimes using the stan- dard Keldysh nonequilibrium Green’s function technique.

\begin{thebibliography}{10}
\bibitem{ref1}
Y, K, Kato. Observation of the Spin Hall Effect in Semiconductors[J]. Science, 2004.
\end{thebibliography}

\end{CJK}
\end{document}
