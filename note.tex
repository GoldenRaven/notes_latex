\documentclass[11pt,a4paper]{article}
\usepackage{xcolor}
\usepackage{CJKutf8}
\usepackage{graphicx}
\usepackage{amsmath}
\usepackage{braket}
\usepackage{caption}
\usepackage{bm}
\usepackage{geometry}
\geometry{a4paper,scale=0.8}

\captionsetup{font={scriptsize}}

\begin{CJK}{UTF8}{gbsn}
\begin{document}

\title{Notes on PRB.67.092408}
% \author{Li Gaoyang}
\date{}
% \date{\today}
\maketitle
% \tableofcontents

% \newpage
\section{spin field effect transistor}
A type of quantum field effect transistor that operates purely on the flow of spin current in the absence of charge current.
\begin{thebibliography}{10}
\bibitem{ref1}
Y, K, Kato. Observation of the Spin Hall Effect in Semiconductors[J]. Science, 2004.
\end{thebibliography}

\end{CJK}
\end{document}
