\documentclass[11pt,a4paper]{article}
\usepackage{xcolor}
\usepackage{CJKutf8}
\usepackage{graphicx}
\usepackage{amsmath}
\usepackage{braket}
\usepackage{caption}
\usepackage{bm}
\usepackage{geometry}
\geometry{a4paper,scale=0.9}

\captionsetup{font={scriptsize}}

\begin{CJK}{UTF8}{gbsn}
\begin{document}

\title{Notes on PRB.67.092408}
% \author{Li Gaoyang}
\date{}
% \date{\today}
\maketitle
% \tableofcontents

% \newpage
\section{spin field effect transistor}
A type of quantum field effect transistor that operates purely on the flow of spin current in the absence of charge current. The rotating field induces a time-independent dc spin current, and at the same time generates no charge cur- rent. The physical principle of our SFET is due to a spin flip mecha- nism provided by the field.
\section{Hamiltonian}
A rotating magnetic field is
\begin{equation}
B_{x} = B_{0}\rm{sin}\theta ~\rm{cos}(\omega t)
\end{equation}
\begin{equation}
B_{y} = B_{0}\rm{sin}\theta~\rm{sin}(\omega t)
\end{equation}
\begin{equation}
B_{z} = B_{0}\rm{cos}\theta.
\end{equation}
The Hamiltonian of system is
\begin{equation}
\begin{split}
H=&\sum_{k, \sigma, \alpha=L, R} \epsilon_{k} C_{k \alpha \sigma}^{+} C_{k \alpha \sigma}+\sum_{n\sigma}\left[\epsilon_{n}+\sigma B_{0} \cos \theta\right] d_{n\sigma}^{+} d_{n\sigma}\\
&+H^{\prime}(t)+\sum_{k,n, \sigma, \alpha=L, R}\left[T_{k \alpha n} C_{k \alpha \sigma}^{+} d_{n\sigma}+\mathrm{c.c.}\right]
\end{split}
\end{equation}
We assume that there are multiple orbits in the scattering region, which is different from the original paper, in which only one orbit is considered. The level energy is
\begin{equation}
\epsilon_{n}(t) = \epsilon_{n}^{0} -qV_{g}(t),
\end{equation}
which can be controlled by the gate voltage $V_{g}$. Energy level of lead $\epsilon_{Lk} = \epsilon_{Rk} = \epsilon_{k}$.

A counterclock-wise rotating field allows a spin-down electron to absorb a photon and flip to spin-up, and it does not allow a spin-up electron to absorb a photon and flip to spin-down.
\begin{equation}
H^{\prime}(t)=\sum_{n}\gamma\left[\exp (-i \omega t) d_{n\uparrow}^{+} d_{n\downarrow}+\exp (i \omega t) d_{n\downarrow}^{+} d_{n\uparrow}\right]
\end{equation}
\begin{equation}
\gamma=B_{0} \sin \theta
\end{equation}
We solve the transport properties (charge and spin currents) of the model in both adiabatic and nonadiabatic regimes using the standard Keldysh nonequilibrium Green’s function technique.
\section{Operator evolution}
EoM of $d_{\sigma}^{\dag}$ is
\begin{equation}
\partial_{t'} d_{n\sigma}^{\dag}(t') = i[H, d_{n\sigma}^{\dag}]
\label{eq:eom1}
\end{equation}
central dot part:
\begin{equation}
[\sum_{n'\sigma'}\left[\epsilon_{n'}+\sigma B_{0} \cos \theta\right] d_{n'\sigma'}^{\dag} d_{n'\sigma'}, d_{n\sigma}^{\dag}] =(\epsilon_{n}+\sigma B_{0} \cos \theta) d_{n\sigma}^{\dag}
\label{eq:part1}
\end{equation}
rotating magnetic field part:
\begin{equation}
\begin{split}
[d_{n'\uparrow}^{\dag} d_{n'\downarrow}, d_{n\sigma}^{\dag}] &= d_{n'\uparrow}^{\dag}\{d_{n'\downarrow}, d_{n\sigma}^{\dag}\} - \{d_{n'\uparrow}^{\dag}, d_{n\sigma}^{\dag}\}d_{n'\downarrow} \\
&=d_{\uparrow}^{\dag}\delta_{nn'}\delta_{\sigma\downarrow},
\end{split}
\end{equation}
\begin{equation}
\begin{split}
[d_{n'\downarrow}^{\dag} d_{n'\uparrow}, d_{n\sigma}^{\dag}] &= d_{n'\downarrow}^{\dag}\{d_{n'\uparrow}, d_{n\sigma}^{\dag}\} - \{d_{n'\downarrow}^{\dag}, d_{n\sigma}^{\dag}\}d_{n'\uparrow} \\
&=d_{n'\uparrow}^{\dag}\delta_{nn'}\delta_{\sigma\uparrow}.
\end{split}
\end{equation}
Then
\begin{equation}
[H^{\prime}(t), d_{n\sigma}^{\dag}] = \gamma(e^{-i \omega t} d_{n\uparrow}^{\dag} \delta_{\sigma\downarrow} + e^{i \omega t} d_{n\downarrow}^{\dag} \delta_{\sigma\uparrow})
\end{equation}
Coupling part
\begin{equation}
\sum_{k, n',\sigma', \alpha=L, R}\left[T_{n'k \alpha} C_{k \alpha \sigma'}^{\dag} d_{n'\sigma'}+\mathrm{c.c.}, d_{n\sigma}^{\dag}\right] = \sum_{k, \alpha=L, R}T_{k \alpha n} C_{k \alpha \sigma}^{\dag}
\label{eq:part3}
\end{equation}
Substitute these three parts into Eq. (\ref{eq:eom1}), we get
\begin{equation}
\partial_{t'} d_{n\sigma}^{\dag}(t') = i[(\epsilon_{n}^{0}+\sigma B_{0} \cos \theta) d_{n\sigma}^{\dag} + \gamma(e^{-i \omega t} d_{n\uparrow}^{\dag} \delta_{\sigma\downarrow} + e^{i \omega t} d_{n\downarrow}^{\dag} \delta_{\sigma\uparrow}) + \sum_{k, \alpha=L, R}T_{k \alpha n} C_{k \alpha \sigma}^{\dag}]
\end{equation}
Equation of motion of particle operator $\hat{N}_{\alpha k\sigma}$ in the lead $\alpha$ is
\begin{equation}
\begin{split}
\frac{d}{dt}\hat{N}_{\alpha k\sigma} &= \frac{i}{\hbar}[H, C_{\alpha k\sigma}^{\dag}C_{\alpha k\sigma}] = \left[\sum_{k', \sigma', \alpha'=L, R}\left[T_{k' \alpha'} C_{k' \alpha' \sigma'}^{+} d_{\sigma'}+\mathrm{c.c.}\right], C_{\alpha k\sigma}^{\dag}C_{\alpha k\sigma}\right]\\
&=\frac{i}{\hbar}\sum_{k', \sigma', \alpha'=L, R}\left[ -T_{k' \alpha'} C_{k' \alpha' \sigma'}^{+} d_{\sigma'}\delta_{\alpha\alpha'}\delta_{kk'}\delta_{\sigma\sigma'}+\mathrm{c.c.}\right]\\
&=\frac{i}{\hbar}[-T_{k \alpha} C_{k \alpha \sigma}^{\dag} d_{\sigma} + T_{k \alpha}^{*} d_{\sigma}^{\dag}C_{k \alpha \sigma}]
\end{split}
\end{equation}
\section{Charge current }
So, the charge current due to L(R) lead with spin $\sigma$ is given by
\begin{equation}
\begin{split}
I_{L\sigma}(t) &= e\langle\frac{d}{dt}\hat{N}_{\sigma}(t)\rangle \\
&=\frac{ie}{\hbar}\sum_{kn\alpha\in L}(\langle -T_{k \alpha n} C_{k \alpha \sigma}^{\dag}(t) d_{n\sigma}(t)\rangle + \langle T_{k \alpha n}^{*} d_{n\sigma}^{\dag}(t)C_{k \alpha \sigma}(t)\rangle)
\end{split}
\end{equation}
Define the lesser Green's function
\begin{equation}
G_{n\sigma',k\alpha\sigma}^{<}(\tau,\tau') = i\langle C_{k\alpha\sigma}^{\dag}(\tau') d_{n\sigma'}(\tau)\rangle ,
\label{eq:contour}
\end{equation}
the current becomes
\begin{equation}
I_{L\sigma}(t)=\frac{-e}{\hbar}\sum_{kn\alpha\in L}(T_{k\alpha n}G_{n,k\alpha\sigma}^{<}(t,t) - T_{k \alpha n}^{*} G_{k\alpha,n}^{<}(t,t)\rangle)
\end{equation}
More generally, we define the contour Green's function
\begin{equation}
G_{n\sigma',k\alpha\sigma}(\tau,\tau') = -i\langle  d_{n\sigma'}(\tau) C_{k\alpha\sigma}^{\dag}(\tau')\rangle .
\end{equation}
EoM of operator $C_{k\alpha \sigma}^{\dag}$ is
\begin{equation}
\partial_{t'} C_{k\alpha \sigma}^{\dag}(t') = i[H, C_{k\alpha \sigma}^{\dag}] = i( \varepsilon_{k}C_{k\alpha \sigma}^{\dag} + \sum_{n}T_{k\alpha n}d_{n\sigma}^{\dag})
\end{equation}
The equation-of-motion for the time-ordered Green function
\begin{equation}
\begin{split}
-i \frac{\partial}{\partial t^{\prime}} G_{n\sigma',k\alpha\sigma}^{t} \left(t,t^{\prime}\right)&= \delta(t-t')\langle \{d_{n\sigma'}, C_{k\alpha\sigma}^{\dag}\}\rangle - \langle T_{c}d_{n\sigma'}\partial_{t'}C_{k\alpha\sigma}^{\dag}\rangle \\
&=\varepsilon_{k} G_{n\sigma',k\alpha\sigma}^{t}\left(t,t^{\prime}\right)+ \sum_{m} T_{k\alpha m}^{*} G_{n\sigma',m\sigma}^{t}(t,t^{\prime})
\end{split}
\end{equation}
So, we have
\begin{equation}
(-i \frac{\partial}{\partial t^{\prime}} -\varepsilon_{k})G_{n\sigma',k\alpha\sigma}^{t} \left(t,t^{\prime}\right)= \sum_{m} T_{k\alpha m}^{*} G_{n\sigma',m\sigma}^{t}(t,t^{\prime})
\label{eq:free}
\end{equation}
in which
\begin{equation}
G_{n\sigma',m\sigma}^{t}(t,t^{\prime}) = -i\langle T_{c}d_{n\sigma'}(t)d_{m\sigma}^{\dag}(t')\rangle.
\end{equation}
Similarly, we evaluate the EoM for free Green's function $g_{k\alpha\sigma}^{t}(t,t')$ in lead $\alpha$ (note that $H=\sum_{k\sigma\alpha} \epsilon_{k} C_{k \alpha \sigma}^{\dag} C_{k \alpha \sigma}$).
\begin{equation}
\begin{split}
-i \frac{\partial}{\partial t^{\prime}} g_{k\alpha\sigma}^{t} \left(t,t^{\prime}\right)&= \delta(t-t')\langle \{C_{k\alpha\sigma}, C_{k\alpha\sigma}^{\dag}\}\rangle - \langle T_{c}C_{k\alpha\sigma}\partial_{t'}C_{k\alpha\sigma}^{\dag}\rangle \\
&=\delta(t-t') + \varepsilon_{k} g_{k\alpha\sigma}^{t}\left(t,t^{\prime}\right),
\end{split}
\end{equation}
we have
\begin{equation}
(-i \frac{\partial}{\partial t^{\prime}}-\varepsilon_{k}) g_{k\alpha\sigma}^{t} \left(t,t^{\prime}\right)=\delta(t-t').
\label{eq:free-delta}
\end{equation}
Substitute Eq. (\ref{eq:free-delta}) into Eq. (\ref{eq:free}) and integrate on both sides, we get an equation analogous to Jauho's notation~\cite{Jauho},
\begin{equation}
G_{n,k\alpha}(\tau,\tau')=\sum_{m} \int d \tau_{1} G_{n m}\left(\tau, \tau_{1}\right) t_{k \alpha m}^{*} g_{k \alpha}\left(\tau_{1}, \tau^{\prime}\right),
\nonumber
\end{equation}
we have
\begin{equation}
G_{n\sigma',k\alpha\sigma}^{t}(t,t')=\sum_{m}\int d t_{1} G_{n\sigma',m\sigma}\left(t, t_{1}\right) T_{k \alpha m}^{*} g_{k \alpha\sigma}^{t}\left(\tau_{1}, \tau^{\prime}\right).
\end{equation}
When there is only one orbit presents, this equation reduces to
\begin{equation}
G_{\sigma',k\alpha\sigma}^{t}(t,t')=\int d t_{1} G_{\sigma',\sigma}\left(t, t_{1}\right) T_{k \alpha}^{*} g_{k \alpha\sigma}^{t}\left(\tau_{1}, \tau^{\prime}\right).
\end{equation}
Since the contour Green's function has the same structure as real-time Green's function, the we have relation
\begin{equation}
G_{n\sigma',k\alpha\sigma}(\tau,\tau')=\sum_{m}\int d \tau_{1} G_{n\sigma',m\sigma}\left(\tau, \tau_{1}\right) T_{k \alpha m}^{*} g_{k \alpha\sigma}\left(\tau_{1}, \tau^{\prime}\right)
\end{equation}
where $G_{n\sigma',k\alpha\sigma}(\tau,\tau')$ is contour Green's function defined in Eq. (\ref{eq:contour}), and similarly the contour Green's function for non-interacting lead is defined as
\begin{equation}
g_{k\alpha\sigma}(\tau,\tau') = -i\langle T_{c}C_{k\alpha\sigma}(\tau)C_{k\alpha\sigma}^{\dag}(\tau')\rangle
\end{equation}
After analytic continuation, the current is formulated as
\begin{equation}
\begin{aligned}
I_{\alpha\sigma}(t) &=-\frac{e}{\hbar} \int d t_{1} \operatorname{Tr}\left[G^{r}\left(t, t_{1}\right) \Sigma_{\alpha}^{<}\left(t_{1}, t\right)+G^{<}\left(t, t_{1}\right) \Sigma_{\alpha}^{a}\left(t_{1}, t\right)+c.c.\right] \\
% &= -\frac{e}{\hbar} \int d t_{1} \operatorname{Tr}\left[G^{r}\left(t, t_{1}\right) \Sigma_{\alpha}^{<}\left(t_{1}, t\right) + G^{<}\left(t, t_{1}\right) \Sigma_{\alpha}^{a}\left(t_{1}, t\right)\\
% &\qquad\qquad\qquad +G^{r*}\left(t, t_{1}\right) \Sigma_{\alpha}^{<*}\left(t_{1}, t\right) + G^{<*}\left(t, t_{1}\right) \Sigma_{\alpha}^{a*}\left(t_{1}, t\right) ] \\
&= -\frac{e}{\hbar} \int d t_{1} \operatorname{Tr}\left[G^{r}\left(t, t_{1}\right) \Sigma_{\alpha}^{<}\left(t_{1}, t\right) + G^{<}\left(t, t_{1}\right) \Sigma_{\alpha}^{a}\left(t_{1}, t\right)\\
&\qquad\qquad\qquad +G^{a}\left(t_{1},t\right) (-1)\Sigma_{\alpha}^{<}\left(t, t_{1}\right) + (-1)G^{<}\left(t_{1}, t\right) \Sigma_{\alpha}^{r}\left(t,t_{1}\right) ] \\
&= -\frac{e}{\hbar} \int d t_{1} \operatorname{Tr}\left[G^{r}\left(t, t_{1}\right) \Sigma_{\alpha}^{<}\left(t_{1}, t\right) + G^{<}\left(t, t_{1}\right) \Sigma_{\alpha}^{a}\left(t_{1}, t\right)\\
&\qquad\qquad\qquad - \Sigma_{\alpha}^{<}\left(t, t_{1}\right)G^{a}\left(t_{1},t\right) -  \Sigma_{\alpha}^{r}\left(t,t_{1}\right) G^{<}\left(t_{1}, t\right) ] 
\end{aligned}
\end{equation}
Following Eq. (224) in WangJ's note, its Fourier transformation is
\begin{equation}
\begin{split}
I_{\alpha\sigma}(\omega_{1})&=-\frac{e}{\hbar}\int \frac{d E}{2 \pi} \frac{d E^{\prime}}{2 \pi} \rm{Tr}\left[G^{r}\left(E+\omega_{1}, E^{\prime}\right) \Sigma_{\alpha}^{<}\left(E^{\prime}, E\right) + G^{<}(E+\omega_{1}, E')\Sigma_{\alpha}^{a}(E', E) + c.c.\right]\\
&=-\frac{e}{\hbar}\int \frac{d E}{2 \pi} \frac{d E^{\prime}}{2 \pi} \rm{Tr} [ G^{r}\left(E+\omega_{1}, E^{\prime}\right) \Sigma_{\alpha}^{<}\left(E^{\prime}, E\right) + G^{<}(E+\omega_{1}, E')\Sigma_{\alpha}^{a}(E', E) \\
& \qquad \qquad \qquad\qquad\quad  -\Sigma_{\alpha}^{<}\left(E+\omega_{1}, E'\right) G^{a}\left(E', E\right)-\Sigma_{\alpha}^{r}\left(E+\omega_{1}, E'\right) G^{<} \left(E', E\right) ] \\
% &= -\frac{e}{\hbar}\int \frac{d E}{2 \pi} \frac{d E^{\prime}}{2 \pi} \rm{Tr} [ G^{r}\left(E+\omega_{1}, E^{\prime}\right) \Sigma_{\alpha}^{<}\left(E^{\prime}, E\right) + G^{<}(E+\omega_{1}, E')\Sigma_{\alpha}^{a}(E', E) \\
% & \qquad \qquad \qquad\qquad\quad  -\Sigma_{\alpha}^{<}\left(E+\omega_{1}, E'\right) G^{a}\left(E', E\right)-\Sigma_{\alpha}^{r}\left(E+\omega_{1}, E'\right) G^{<} \left(E', E\right) ] .
\end{split}
\label{eq:current3}
\end{equation}
Here, $G^{r,<} \equiv G_{n\sigma',m\sigma}^{r,<}$, notation Tr means sum over QD level index $n$, and matrix element
\begin{equation}
\Sigma_{\alpha,mn}^{\gamma}(t_{1}, t_{2}) = \sum_{k} T_{k\alpha m}^{*}(t_{1}) g_{k\alpha}^{\gamma}(t_{1}, t_{2}) T_{k\alpha n}(t_{2}),
\label{eq:sigma}
\end{equation}
and
\begin{equation}
\Sigma_{\alpha, m n}^{\gamma}\left(t_{1}, t_{2}\right)=\sum_{k} T_{k \alpha m}^{*}\left(t_{1}\right) g_{k \alpha}^{\gamma}\left(t_{1}, t_{2}\right) T_{k \alpha n}\left(t_{2}\right)
\end{equation}
in which, $g_{k\sigma}$ is the free propagator of lead, and $\gamma$ is \{>, <, r, a\}. From Keldysh equation, we have (different from Eq. (33) in Chap. II ?)
\begin{equation}
G^{<}=G^{r} \Sigma^{<} G^{a}
\label{eq:r<a}
\end{equation}
whose Fourier transformation gives
\begin{equation}
G^{<}(E_{1},E_{2})=\int\frac{dE}{2\pi}\frac{dE'}{2\pi}G^{r}(E_{1}, E) \Sigma^{<}(E, E') G^{a}(E', E_{2})
\end{equation}
\section{Calculate $G^{r}$}
For $G^{r}$, the Dyson equation gives (refer to \cite{Baigeng})
\begin{equation}
G^{r}(t_{1},t_{2})=G^{0r}(t_{1}-t_{2})+\int dt ~G^{0r}(t_{1}-t) H'(t) G^{r}(t,t_{2})
\label{eq:dyson-2}
\end{equation}
in which, $G^{0r}$ is not the free propagator of central dot, but the equilibrium Green’s function when the pumping potential H′ is set to zero, i.e. the hamiltonian is $H-H'$, not $\varepsilon_{n}d_{n\sigma}^{\dag}d_{n\sigma}$. Multiply $e^{i E_{1} t_{1}-i E_{2} t_{2}}$ and integrate on both sides, we have
\begin{equation}
\begin{split}
G^{r}(E_{1}, E_{2})\equiv& \int d t_{1} d t_{2} e^{i E_{1} t_{1}-i E_{2} t_{2}} G^{r}(t_{1}, t_{2}) \\
=& 2\pi G^{0r}(E_{1})\delta(E_{1}-E_{2}) \\
&+ \int dt_{1}G^{0r}(t_{1}-t) e^{iE_{1}(t_{1}-t)} H'(t) e^{i(E_{1}-E)t} \iint dtdt_{2} G^{r}(t,t_{2}) e^{iEt-E_{2}t_{2}} \\
=& 2\pi G^{0r}(E_{1})\delta(E_{1}-E_{2}) + \int \frac{dE}{2\pi}G^{0r}(E_{1}) H'(E_{1}-E) G^{r}(E,E_{2}),
\end{split}
\label{eq:Gr}
\end{equation}
in which, we inserted the inverse Fourier transformation
\begin{equation}
H^{\prime}(t) = \int \frac{dE}{2\pi} e^{-i(E_{1}-E)t} H'(E_{1}-E).
\end{equation}
\subsection{Calcualte $G^{0r}$}
Using Eq. (\ref{eq:part1}) and Eq. (\ref{eq:part3}), we have
\begin{equation}
\partial_{t}d_{n\sigma}^{\dag} =i(\epsilon_{n}+\sigma B_{0} \cos \theta) d_{n\sigma}^{\dag} + i\sum_{k, \alpha=L, R}T_{k \alpha n} C_{k \alpha \sigma}^{\dag}
\end{equation}
\begin{equation}
\partial_{t}d_{n\sigma} =-i(\epsilon_{n}+\sigma B_{0} \cos \theta) d_{n\sigma} - i\sum_{k, \alpha=L, R}T_{k \alpha n}^{*} C_{k \alpha \sigma}^{\dag}
\end{equation}
then
\begin{equation}
d_{n\sigma}^{\dag}(t) =d_{n\sigma}^{\dag}(0) e^{i(\epsilon_{n}+\sigma B_{0} \cos \theta)t}  + it\sum_{k, \alpha=L, R}T_{k \alpha n} C_{k \alpha \sigma}^{\dag}
\end{equation}
\begin{equation}
d_{n\sigma}(t) =d_{n\sigma}(0)e^{-i(\epsilon_{n}+\sigma B_{0} \cos \theta)} - it\sum_{k, \alpha=L, R}T_{k \alpha n}^{*} C_{k \alpha \sigma}
\end{equation}
Note $\epsilon_{n}$ is time-dependent, $G^{0r}(t,t')$ depends only on time difference. 
\begin{equation}
\begin{split}
G_{n\sigma}^{0r}(t,t') =& -i\theta(t-t')\langle \{d_{n\sigma}(t),d_{n\sigma}^{\dag}(t')\} \rangle \\
% =&-i\theta(t-t') e^{-i(\epsilon_{n}+\sigma B_{0} \cos \theta) (t-t')}
=&?
\end{split}
\end{equation}
% \begin{equation}
% g^{r}\left(E_{1}, E_{2}\right) = 2\pi g^{r}\left(E_{1}\right) \delta(E_{1}-E_{2})
% \end{equation}
% and
% \begin{equation}
% g_{n\sigma}^{r}(E) = \frac{1}{E-(\epsilon_{n}+\sigma B_{0} \cos \theta)+i 0^{+}}
% \end{equation}
The hamiltonian $H-H'$ cannot flip spin, so in the spin space \{$d_{1,\uparrow}^{\dag}$, $d_{1,\downarrow}^{\dag}, d_{2\uparrow}^{\dag}, d_{2\downarrow}^{\dag}, \cdots$; $d_{1\uparrow}$, $d_{1\downarrow}, d_{2\uparrow}, d_{2\downarrow}, \cdots$\}, $G^{0r}$ is diagonal
\begin{equation}
G^{0r}(E)=\bigotimes_{n}G_{n}^{0r}(E)
\end{equation}
\begin{equation}
G_{n}^{0r}(E) = \left(\begin{array}{cc}
G_{n\uparrow}^{0r}(E) &0 \\
0 & G_{n\downarrow}^{0r}(E)
\end{array}\right).
\end{equation}
\subsection{Calcualte $G^{r}$}
In spin space
\begin{equation}
G^{r}\equiv\left(\begin{array}{cc}
G_{n\uparrow,m\uparrow}^{r} & G_{n\uparrow,m\downarrow}^{r} \\
G_{n\downarrow,m\uparrow}^{r} & G_{n\downarrow,m\downarrow}^{r}
\end{array}\right)
\label{eq:Gr2}
\end{equation}
is a matrix of $2N$ dimension, where $N$ is the total number of levels in central area. $H'$ in Eq. (\ref{eq:dyson-2})is a matrix of same dimension, given by
\begin{equation}
H^{\prime}= \bigotimes_{n} H'_{n},
\end{equation}
\begin{equation}
H_{n}^{\prime}=\left(\begin{array}{cc}
0 & \gamma e^{-i \omega t} \\
\gamma e^{i \omega t} & 0
\end{array}\right).
\end{equation}
For simplicity, we consider only one energy level and neglect level index. we have
\begin{equation}
\begin{split}
H^{\prime}(E_{1}-E)&= \int dt e^{i(E_{1}-E)t} H'(t)\\
&=\left(\begin{array}{cc}
0 & \gamma\int dt e^{i(E_{1}-E-\omega)t}\\
\gamma\int dt e^{i(E_{1}-E+\omega)t} & 0
\end{array}\right) \\
&=2\pi\gamma\left(\begin{array}{cc}
0 & \delta(E_{1}-E-\omega)\\
\delta(E_{1}-E+\omega) & 0
\end{array}\right)
\end{split}
\end{equation}
Substitute these equations into Eq. (\ref{eq:Gr}), we get
\begin{equation}
\begin{split}
\left(\begin{array}{cc}
G_{\uparrow,\uparrow}^{r}(E_{1},E_{2}) & G_{\uparrow,\downarrow}^{r}(E_{1},E_{2}) \\
G_{\downarrow,\uparrow}^{r}(E_{1},E_{2}) & G_{\downarrow,\downarrow}^{r}(E_{1},E_{2})
\end{array}\right) = &
\left(\begin{array}{cc}
G_{\uparrow}^{0r}(E_{1}) &0 \\
0 & G_{\downarrow}^{0r}(E_{1})
\end{array}\right) 2\pi\delta(E_{1}-E_{2})\\
&+ \gamma\int dE\left(\begin{array}{cc}
G_{\uparrow}^{0r}(E_{1}) &0 \\
0 & G_{\downarrow}^{0r}(E_{1})
\end{array}\right) \\
&\times
\left(\begin{array}{cc}
0 & \delta(E_{1}-E-\omega) \\
\delta(E_{1}-E+\omega) & 0
\end{array}\right)
 \left(\begin{array}{cc}
G_{\uparrow,\uparrow}^{r} & G_{\uparrow,\downarrow}^{r} \\
G_{\downarrow,\uparrow}^{r} & G_{\downarrow,\downarrow}^{r}
\end{array}\right) \\
=&\left(\begin{array}{cc}
G_{\uparrow}^{0r}(E_{1}) &0 \\
0 & G_{\downarrow}^{0r}(E_{1})
\end{array}\right) 2\pi\delta(E_{1}-E_{2})\\
&+ \left(\begin{array}{cc}
G_{1}(E_{1},E_{2}) & G_{2}(E_{1},E_{2}) \\
G_{3}(E_{1},E_{2}) & G_{4}(E_{1},E_{2})
\end{array}\right).
\end{split}
\end{equation}
where we have omitted the independent variables $(E, E_{2})$ of $G_{n \sigma, m\sigma'}^{r}$ for sake of convenience. In the above equations $G_{1}, G_{2}, G_{3}, G_{4}$ are
\begin{equation}
G_{1}(E_{1},E_{2}) = \gamma \int dE G_{ \uparrow}^{0r}\left(E_{1}\right) \delta(E_{1}-E-\omega) G_{ \downarrow, \uparrow}^{r}(E,E_{2})
\end{equation}
\begin{equation}
G_{2}(E_{1},E_{2}) = \gamma\int dE G_{ \uparrow}^{0r} \left(E_{1}\right)  \delta(E_{1}-E-\omega) G_{ \downarrow, \downarrow}^{r}(E,E_{2})
\end{equation}
\begin{equation}
G_{3}(E_{1},E_{2}) =  \gamma \int dE G_{ \downarrow}^{0r}\left(E_{1}\right)  \delta(E_{1}-E+\omega) G_{ \uparrow, \uparrow}^{r}(E,E_{2})
\end{equation}
\begin{equation}
G_{4}(E_{1},E_{2}) = \gamma\int dE G_{ \downarrow}^{0r}\left(E_{1}\right)  \delta(E_{1}-E+\omega) G_{ \uparrow, \downarrow}^{r}(E,E_{2})
\end{equation}
So, the $G_{1}$ term
\begin{equation}
\begin{split}
G_{ \uparrow, \uparrow}^{r}(E_{1}, E_{2}) &= G_{ \uparrow}^{0r}\left(E_{1}\right) 2 \pi \delta\left(E_{1}-E_{2}\right) + G_{1}\left(E_{1}, E_{2}\right) \\
&= 2 \pi G_{ \uparrow}^{0r}\left(E_{1}\right) \delta\left(E_{1}-E_{2}\right) + \gamma G_{ \uparrow}^{0r}\left(E_{1}\right)  G_{ \downarrow, \uparrow}^{r}\left(E_{1}-\omega, E_{2}\right)
\end{split}
\end{equation}
The $G_{2}$ term
\begin{equation}
\begin{split}
G_{ \uparrow, \downarrow}^{r}\left(E_{1}, E_{2}\right) =& G_{2}\left(E_{1}, E_{2}\right) \\
=& \gamma G_{ \uparrow}^{0r} \left(E_{1}\right) G_{ \downarrow, \downarrow}^{r}(E_{1}-\omega,E_{2})
\end{split}
\end{equation}
The $G_{3}$ term
\begin{equation}
\begin{split}
G_{ \downarrow, \uparrow}^{r}\left(E_{1}, E_{2}\right) &= G_{3}\left(E_{1}, E_{2}\right) \\
&= \gamma G_{ \downarrow}^{0r} \left(E_{1}\right) G_{ \uparrow, \uparrow}^{r}(E_{1}+\omega,E_{2})
\end{split}
\end{equation}
and the $G_{4}$ term
\begin{equation}
\begin{split}
G_{ \downarrow, \downarrow}^{r}\left(E_{1}, E_{2}\right) &= g_{ \downarrow}^{r}\left(E_{1}\right) 2 \pi \delta\left(E_{1}-E_{2}\right) +G_{4}\left(E_{1}, E_{2}\right)\\
&= 2 \pi G_{ \downarrow}^{0r}\left(E_{1}\right) \delta\left(E_{1}-E_{2}\right) + \gamma G_{ \downarrow}^{0r}(E_{1}) G_{ \uparrow, \downarrow}^{r}(E_{1}+\omega, E_{2})
\end{split}
\end{equation}
After collecting terms, we get
\begin{equation}
\begin{split}
G_{ \uparrow, \uparrow}^{r}(E_{1}, E_{2}) &= 2 \pi G_{ \uparrow}^{0r}\left(E_{1}\right) \delta\left(E_{1}-E_{2}\right) + \gamma G_{ \uparrow}^{0r}\left(E_{1}\right) \gamma G_{\downarrow}^{0r}(E_{1}-\omega)G_{ \uparrow, \uparrow}^{r}\left(E_{1}, E_{2}\right) \\
&=\frac{2 \pi G_{ \uparrow}^{0r}\left(E_{1}\right) \delta\left(E_{1}-E_{2}\right)}{1-\gamma^{2} G_{ \uparrow}^{0r}\left(E_{1}\right) G_{ \downarrow}^{0r}\left(E_{1}-\omega\right)}
\end{split}
\end{equation}
and
\begin{equation}
G_{ \downarrow, \downarrow}^{r}(E_{1}, E_{2}) = \frac{2 \pi G_{ \downarrow}^{0r}\left(E_{1}\right) \delta\left(E_{1}-E_{2}\right)}{1-\gamma^{2} G_{ \downarrow}^{0r}\left(E_{1}\right) G_{ \uparrow}^{0r}\left(E_{1}+\omega\right)}
\end{equation}
and
\begin{equation}
\begin{split}
G_{ \uparrow, \downarrow}^{r}\left(E_{1}, E_{2}\right) &=  \gamma G_{ \uparrow}^{0r} \left(E_{1}\right)  G_{ \downarrow, \downarrow}^{r}(E_{1}-\omega,E_{2}) \\
&=\frac{2 \pi \gamma G_{ \uparrow}^{0r}\left(E_{1}\right) G_{ \downarrow}^{0r}\left(E_{1}-\omega\right) \delta\left(E_{1}-\omega-E_{2}\right) }{1-\gamma^{2} G_{ \downarrow}^{0r}\left(E_{1}-\omega\right) G_{ \uparrow}^{0r}\left(E_{1}\right) }
\end{split}
\end{equation}
\begin{equation}
\begin{split}
G_{ \downarrow, \uparrow}^{r}\left(E_{1}, E_{2}\right) &= \gamma G_{ \downarrow}^{0r} \left(E_{1}\right) G_{ \uparrow, \uparrow}^{r}(E_{1}+\omega,E_{2}) \\
&=\frac{2 \pi\gamma G_{ \uparrow}^{0r}\left(E_{1}+\omega\right) G_{ \downarrow}^{0r} \left(E_{1}\right) \delta(E_{1}+\omega-E_{2}) }{1-\gamma^{2} G_{ \uparrow}^{0r}\left(E_{1}+\omega\right) G_{ \downarrow}^{0r}\left(E_{1}\right)}
\end{split}
\end{equation}
Define
\begin{equation}
g_{\sigma}^{r,a}(E) \equiv \frac{ G_{\sigma}^{0r,a}(E)} {1-\gamma^{2} G_{\sigma}^{0r,a}(E) G_{\bar{\sigma}}^{0r,a}(E+\bar{\sigma}\omega)},
\end{equation}
and denote
\begin{equation}
G^{r}(E_{1},E_{2})=\left(\begin{array}{cc}
G_{11}^{r}(E_{1},E_{2}) & G_{12}^{r}(E_{1},E_{2}) \\
G_{21}^{r}(E_{1},E_{2}) & G_{22}^{r}(E_{1},E_{2})
\end{array}\right)
\end{equation}
then we can write
\begin{equation}
G_{11}^{r}(E_{1},E_{2}) = 2\pi g_{\uparrow}^{r}(E_{1}) \delta(E_{1}-E_{2}).
\end{equation}
\begin{equation}
G_{22}^{r}(E_{1},E_{2}) = 2\pi g_{\downarrow}^{r}(E_{1}) \delta(E_{1}-E_{2}).
\end{equation}
\begin{equation}
G_{12}^{r}(E_{1},E_{2}) = 2\pi \gamma g_{\uparrow}^{r}(E_{1}) G_{\downarrow}^{0r}(E_{1}-\omega) \delta(E_{1}-\omega-E_{2}).
\end{equation}
\begin{equation}
G_{21}^{r}(E_{1},E_{2}) = 2\pi\gamma g_{\downarrow}^{r}(E_{1}) G_{\uparrow}^{0r}(E_{1}+\omega) \delta(E_{1}+\omega-E_{2}).
\end{equation}
Thus we get $G^{r}$ in Eq. (\ref{eq:Gr2}).
\section{Calculate $G^{<}$}
\subsection{Calcualte $\Sigma_{\alpha}$}
In Eq. (\ref{eq:sigma}), 
\begin{equation}
\Sigma_{\alpha}^{\gamma}\left(t_{1}, t_{2}\right)=\sum_{k} T_{k \alpha}^{*}\left(t_{1}\right) g_{k \alpha}^{\gamma}\left(t_{1}, t_{2}\right) T_{k \alpha}\left(t_{2}\right).
\end{equation}
Here we consider parametric pumping, thus no bias presents in the leads as demonstrated in the system Hamiltonian, i.e. $\Sigma_{a}(t_{1},t_{2}) \rightarrow \Sigma_{a} (t_{1}-t_{2})$. Using free propagators $g_{k\alpha\sigma}^{\gamma}$, Fourier-transform to
\begin{equation}
\Sigma_{\alpha\sigma}^{<}(E_{1},E_{2}) =2\pi \Sigma_{\alpha\sigma}^{<}(E_{1})\delta(E_{1}-E_{2}),
\end{equation}
in which
\begin{equation}
\Sigma_{\alpha\sigma}^{<}(E)=i  f(E) \Gamma_{\alpha}\left(E\right).
\end{equation}
The linewidth function $\Gamma$ is defined as
\begin{equation}
\Gamma_{\alpha}(E) \equiv 2 \pi \sum_{k} T_{k \alpha}^{*} T_{k \alpha} \delta\left(E-\epsilon_{k}\right),
\end{equation}
thus we have $\Sigma_{\sigma}$, which is a number in spin space, not a matrix, since $\Sigma^{<}$ is independent of spin $\sigma$. Similarly,
\begin{equation}
\Sigma_{\alpha\sigma}^{a}(E_{1},E_{2}) =2\pi \Sigma_{\alpha\sigma}^{a}(E_{1})\delta(E_{1}-E_{2}),
\end{equation}
and the retarted(advanced) self-energy is (?)
\begin{equation}
\Sigma_{\alpha\sigma}^{r,a}(E)=\Lambda_{\alpha}(E) \mp \frac{i}{2}\Gamma_{\alpha}(E),
\end{equation}
since
\begin{equation}
\Sigma_{\alpha\sigma}^{a}(E)=[\Sigma_{\alpha\sigma}^{r}(E)]^{*}
\end{equation}
and
\begin{equation}
\begin{split}
\Sigma_{\alpha\sigma}^{a}(E)-\Sigma_{\alpha\sigma}^{r}(E)&=i \Gamma_{\alpha\sigma}(E) \\
& = 2i\rm{Im}\{\Sigma_{\alpha\sigma}^{a}(E)\}.
\end{split}
\end{equation}

\subsection{Calculate $G^{a}$}
We have relation
\begin{equation}
G_{\sigma,\sigma'}^{a}\left(E_{1}, E_{2}\right)=(G_{\sigma',\sigma}^{r}\left(E_{2}, E_{1}\right))^{*},
\end{equation}
\begin{equation}
G^{r}(E_{1},E_{2})=\left(\begin{array}{cc}
G_{11}^{r}(E_{1},E_{2}) & G_{12}^{r}(E_{1},E_{2}) \\
G_{21}^{r}(E_{1},E_{2}) & G_{22}^{r}(E_{1},E_{2})
\end{array}\right)
\end{equation}
so in spin space, 
\begin{equation}
G^{a}(E_{1},E_{2})=\left(\begin{array}{cc}
(G_{11}^{r}(E_{2},E_{1}))^{*} & (G_{21}^{r}(E_{2},E_{1}))^{*} \\
(G_{12}^{r}(E_{2},E_{1}))^{*} & (G_{22}^{r}(E_{2},E_{1}))^{*}
\end{array}\right)
\end{equation}

\subsection{Calculate $G^{<}$}
Substitute $G^{r}, \Sigma^{<}, G^{a}$ into Eq. (\ref{eq:r<a}), we get $G^{<}$,
\begin{equation}
\begin{split}
G^{<}(E_{1},E_{2})&=\int\frac{dE}{2\pi}\frac{dE'}{2\pi}G^{r}(E_{1}, E) \Sigma^{<}(E, E') G^{a}(E', E_{2}) \\
&= \int \frac{dE}{2\pi} G^{r}\left(E_{1}, E\right) \Sigma^{<}\left(E\right) G^{a}\left(E, E_{2}\right) \\
&= \int \frac{dE}{2\pi} if(E)\Gamma(E) G^{r}\left(E_{1}, E\right) G^{a}\left(E, E_{2}\right)
\end{split}
\end{equation}
\begin{equation}
\begin{split}
G^{r}\left(E_{1}, E\right) G^{a}\left(E, E_{2}\right) =
\left(\begin{array}{cc}
G_{1} & G_{2} \\
G_{3} & G_{4}
\end{array}\right)
\end{split}
\end{equation}
in which, the $G_{1}$ term is
\begin{equation}
\begin{split}
G_{1} &= G_{11}^{r}(E_{1}, E)G_{11}^{r*}(E_{2},E)+G_{12}^{r}(E_{1},E)G_{12}^{r*}(E_{2},E) \\
&= (2 \pi)^{2} g_{\uparrow}^{r}\left(E_{1}\right)  g_{\uparrow}^{r*}\left(E_{2}\right) \delta\left(E_{2}-E\right) \delta\left(E_{1}-E\right) \\
&\quad + (2 \pi \gamma)^{2} g_{\uparrow}^{r}\left(E_{1}\right) G_{\downarrow}^{0r}\left(E_{1}-\omega\right)  g_{\uparrow}^{r*}\left(E_{2}\right) G_{\downarrow}^{0r*}\left(E_{2}-\omega\right) \delta\left(E_{1}-\omega-E\right) \delta\left(E_{2}-\omega-E\right)
\end{split}
\end{equation}
the $G_{2}$ term is
\begin{equation}
G_{2} = G_{11}^{r}(E_{1}, E)G_{21}^{r*}(E_{2},E) + G_{12}^{r}(E_{1}, E)G_{22}^{r*}(E_{2},E)
\end{equation}
the $G_{3}$ term is
\begin{equation}
G_{3} = G_{21}^{r}(E_{1},E)G_{11}^{r}(E_{2},E) + G_{22}^{r}(E_{1},E)G_{12}^{r*}(E_{2},E)
\end{equation}
the $G_{4}$ term is
\begin{equation}
\begin{split}
G_{4} &=G_{21}^{r}(E_{1},E)G_{21}^{r*}(E_{2},E) + G_{22}^{r}(E_{1},E)G_{22}^{r*}(E_{2},E) \\
&= (2 \pi)^{2} g_{\downarrow}^{r}\left(E_{1}\right)  g_{\downarrow}^{r*}\left(E_{2}\right) \delta\left(E_{2}-E\right) \delta\left(E_{1}-E\right) \\
&\quad + (2 \pi \gamma)^{2} g_{\downarrow}^{r}\left(E_{1}\right) G_{\uparrow}^{0r}\left(E_{1}+\omega\right)  g_{\downarrow}^{r*}\left(E_{2}\right) G_{\uparrow}^{0r*}\left(E_{2}+\omega\right) \delta\left(E_{1}+\omega-E\right) \delta\left(E_{2}+\omega-E\right)
\end{split}
\end{equation}
So, the matrix element of $G^{<}$ is
\begin{equation}
\begin{split}
G^{<}\left(E_{1}, E_{2}\right) =
\left(\begin{array}{cc}
G_{11}^{<} & G_{12}^{<} \\
G_{21}^{<} & G_{22}^{<}
\end{array}\right)
\end{split}
\end{equation}
\begin{equation}
\begin{split}
G_{11}^{<}(E_{1},E_{2})&=\int \frac{d E}{2 \pi} i f(E) \Gamma(E) (2 \pi)^{2} g_{\uparrow}^{r}\left(E_{1}\right)  g_{\uparrow}^{r,*}\left(E_{2}\right) \delta\left(E_{2}-E\right) \delta\left(E_{1}-E\right) \\
&\quad+\int \frac{d E}{2 \pi} i f(E) \Gamma(E) (2 \pi \gamma)^{2} g_{\uparrow}^{r}\left(E_{1}\right) G_{\downarrow}^{0r}\left(E_{1}-\omega\right)  g_{\uparrow}^{r*}\left(E_{2}\right) G_{\downarrow}^{0r*}\left(E_{2}-\omega\right) \delta\left(E_{1}-\omega-E\right) \delta\left(E_{2}-\omega-E\right)\\
&= 2\pi i f(E_{1}) \Gamma(E_{1}) g_{\uparrow}^{r}\left(E_{1}\right)  g_{\uparrow}^{r,*}\left(E_{2}\right) \delta(E_{2}-E_{1}) \\
&\quad+ 2\pi \gamma^{2} i f(E_{1}-\omega) \Gamma(E_{1}-\omega) g_{\uparrow}^{r}\left(E_{1}\right) G_{\downarrow}^{0r}\left(E_{1}-\omega\right)  g_{\uparrow}^{r*}\left(E_{2}\right) G_{\downarrow}^{0r*}\left(E_{2}-\omega\right) \delta(E_{2}-E_{1})
\end{split}
\end{equation}
\begin{equation}
\begin{split}
G_{22}^{<}(E_{1},E_{2})&= 2\pi i f(E_{1}) \Gamma(E_{1}) g_{\downarrow}^{r}\left(E_{1}\right)  g_{\downarrow}^{r,*}\left(E_{2}\right) \delta(E_{2}-E_{1}) \\
&\quad+ 2\pi \gamma^{2} i f(E_{1}+\omega) \Gamma(E_{1}+\omega) g_{\downarrow}^{r}\left(E_{1}\right) G_{\uparrow}^{0r}\left(E_{1}+\omega\right)  g_{\downarrow}^{r*}\left(E_{2}\right) G_{\uparrow}^{0r*}\left(E_{2}+\omega\right) \delta(E_{2}-E_{1})
\end{split}
\end{equation}
\section{Spin-up current}
The spin-up current (\ref{eq:current3}) is
\begin{equation}
\begin{split}
I_{\alpha\uparrow,\uparrow}(\omega_{1})&=-\frac{e}{\hbar}\int \frac{d E}{2 \pi} \frac{d E^{\prime}}{2 \pi} \big[G_{11}^{r} \left(E+\omega_{1}, E^{\prime}\right) 2\pi if(E')\Gamma_{\alpha}(E')\delta(E^{\prime}- E)  + G_{11}^{<}(E+\omega_{1}, E')2\pi\Sigma_{\alpha}^{a}(E')\delta(E'-E) \\
& \quad -2\pi if(E+\omega_{1})\Gamma_{\alpha}(E+\omega_{1})\delta(E+\omega_{1} - E') G_{11}^{a}\left(E', E\right)-2\pi\Sigma_{\alpha}^{r}\left(E+\omega_{1}\right)\delta(E+\omega_{1}-E') G_{11}^{<} \left(E', E\right) \big] \\
&=-\frac{e}{\hbar}\int \frac{d E}{2 \pi} \big[G_{11}^{r}\left(E+\omega_{1}, E\right) if(E)\Gamma_{\alpha}(E) + G_{11}^{<}(E+\omega_{1}, E)\Sigma_{\alpha}^{a}(E) \\
&\quad - if(E+\omega_{1})\Gamma_{\alpha}(E+\omega_{1}) G_{11}^{a}\left(E+\omega_{1}, E\right) - \Sigma_{\alpha}^{r}\left(E+\omega_{1}\right) G_{11}^{<} \left(E+\omega_{1}, E\right) \big].
\end{split}
\label{eq:current4}
\end{equation}
Recall that
\begin{equation}
G_{11}^{r}(E_{1},E_{2}) = 2\pi g_{\uparrow}^{r}(E_{1}) \delta(E_{1}-E_{2})
\label{eq:Gupup}
\end{equation}
\begin{equation}
G_{11}^{a}(E_{1},E_{2}) = 2\pi g_{\uparrow}^{a}(E_{1}) \delta(E_{1}-E_{2})
\label{eq:Gupupa}
\end{equation}
\begin{equation}
G_{11}^{<}(E_{1},E_{2}) = 2\pi [A_{11}(E_{1},E_{2})+B_{11}(E_{1},E_{2})] \delta(E_{2}-E_{1}),
\label{eq:G11}
\end{equation}
in which
\begin{equation}
A_{11}(E_{1},E_{2})= i f(E_{1}) \Gamma(E_{1}) g_{\uparrow}^{r}\left(E_{1}\right)  g_{\uparrow}^{r,*}\left(E_{2}\right)
\end{equation}
\begin{equation}
B_{11}(E_{1},E_{2}) = \gamma^{2} i f(E_{1}-\omega) \Gamma(E_{1}-\omega) g_{\uparrow}^{r}\left(E_{1}\right) G_{\downarrow}^{0r}\left(E_{1}-\omega\right)  g_{\uparrow}^{r*}\left(E_{2}\right) G_{\downarrow}^{0r*}\left(E_{2}-\omega\right) .
\end{equation}
Substitute Eq. (\ref{eq:Gupup}) (\ref{eq:Gupupa}) and (\ref{eq:G11}) into (\ref{eq:current4}), we have
\begin{equation}
\begin{split}
I_{\alpha\uparrow,\uparrow}(\omega_{1})&= -\frac{e}{\hbar}\int dE g_{\uparrow}^{r}(E+\omega_{1}) if(E)\Gamma_{\alpha}(E)\delta(\omega_{1}) + [A_{11}(E+\omega_{1},E)+B_{11}(E+\omega_{1},E)]\delta(-\omega_{1}) \Sigma_{\alpha}^{a}(E)\\
&\quad -if\left(E+\omega_{1}\right) \Gamma_{\alpha}\left(E+\omega_{1}\right) g_{\uparrow}^{a}\left(E+\omega_{1}\right) \delta(\omega_{1}) - \Sigma_{\alpha}^{r}\left(E+\omega_{1}\right) [A_{11}(E+\omega_{1},E)+B_{11}(E+\omega_{1},E)]\delta(-\omega_{1})
\end{split}
\label{eq:current5}
\end{equation}
Omit $\omega_{1}$, we get a DC current
\begin{equation}
\begin{split}
I_{\alpha\uparrow,\uparrow}&= -\frac{e}{\hbar}\int dE g_{\uparrow}^{r}(E) if(E)\Gamma_{\alpha}(E) + [A_{11}(E)+B_{11}(E)] \Sigma_{\alpha}^{a}(E)\\
&\quad-if\left(E\right) \Gamma_{\alpha}\left(E\right) g_{\uparrow}^{a}\left(E\right)  - \Sigma_{\alpha}^{r}\left(E\right) [A_{11}(E)+B_{11}(E)] \\
&=-\frac{e}{\hbar}\int dE \big[g_{\uparrow}^{r}(E) - g_{\uparrow}^{a}\left(E\right) \big] if(E)\Gamma_{\alpha}(E)  + [A_{11}(E)+B_{11}(E)]i\Gamma_{\alpha}(E),
\end{split}
\label{eq:current6}
\end{equation}
in which,
% \begin{equation}
% g_{\sigma}^{r,a}(E) \equiv \frac{ G_{\sigma}^{0r,a}(E)} {1-\gamma^{2} G_{\sigma}^{0r,a}(E) G_{\bar{\sigma}}^{0r,a}(E+\bar{\sigma}\omega)},
% \end{equation}
\begin{equation}
A_{11}(E)= i f(E) \Gamma(E) |g_{\uparrow}^{r}\left(E\right)|^{2},
\end{equation}
\begin{equation}
B_{11}(E) = \gamma^{2} i f(E-\omega) \Gamma(E-\omega) |g_{\uparrow}^{r}\left(E\right)|^{2} |G_{\downarrow}^{0r}\left(E-\omega\right)|^{2}.
\end{equation}
Using the following relations
\begin{equation}
g_{\uparrow}^{r}(E) = \frac{ G_{\uparrow}^{0r}(E)} {1-\gamma^{2} G_{\uparrow}^{0r}(E) G_{\downarrow}^{0r}(E-\omega)},
\end{equation}
\begin{equation}
g_{\uparrow}^{a}(E) = \frac{ G_{\uparrow}^{0a}(E)} {1-\gamma^{2} G_{\uparrow}^{0a}(E) G_{\downarrow}^{0a}(E-\omega)},
\end{equation}
in which,
\begin{equation}
G_{\uparrow}^{0 r}(E)=\frac{1}{E-\varepsilon^{0}-B_{0} \cos \theta+i \Gamma(E) / 2}
\end{equation}
\begin{equation}
G_{\downarrow}^{0 r}(E-\omega)=\frac{1}{E-\omega-\varepsilon^{0}+B_{0} \cos \theta+i \Gamma(E) / 2},
\end{equation}
note that
\begin{equation}
G_{\uparrow}^{0 r}(E) - G_{\uparrow}^{0 a}(E)=-i\Gamma(E) |G_{\uparrow}^{0 r}(E)|^{2}
\end{equation}
\begin{equation}
G_{\downarrow}^{0 r}(E-\omega) - G_{\downarrow}^{0 a}(E-\omega)=-i\Gamma(E-\omega) |G_{\downarrow}^{0 r}(E-\omega)|^{2},
\end{equation}
we get
\begin{equation}
\begin{split}
g_{\uparrow}^{r}(E) - g_{\uparrow}^{a}(E) &= \frac{G_{\uparrow}^{0r}(E)[1-\gamma^{2} G_{\uparrow}^{0a}(E) G_{\downarrow}^{0a}(E-\omega)]- G_{\uparrow}^{0a}(E)[1-\gamma^{2} G_{\uparrow}^{0r}(E) G_{\downarrow}^{0r}(E-\omega)]} {|1-\gamma^{2} G_{\uparrow}^{0r}(E) G_{\downarrow}^{0r}(E-\omega)|^{2}} \\
&=\frac{-i\Gamma(E) |G_{\uparrow}^{0 r}(E)|^{2}-i\gamma^{2}\Gamma(E-\omega) |G_{\uparrow}^{0r}|^{2}} {|1-\gamma^{2} G_{\uparrow}^{0r}(E) G_{\downarrow}^{0r}(E-\omega)|^{2}},
\end{split}
\end{equation}
which cancels $A_{11}$ term in DC current equation and yields
\begin{equation}
\begin{split}
I_{\alpha\uparrow,\uparrow}&=-\frac{e}{\hbar}\int dE \gamma^{2}\Gamma(E-\omega) \Gamma_{\alpha}(E)|g_{\uparrow}^{r}\left(E\right)|^{2} |G_{\downarrow}^{0r} (E-\omega)|^{2} \big[f(E) - f(E-\omega) \big],
\end{split}
\label{eq:current6}
\end{equation}
\section{Spin-down current}
From Eq. (\ref{eq:current6}) we have
\begin{equation}
I_{\alpha\downarrow,\downarrow}= -\frac{e}{\hbar}\int dE \gamma^{2}\Gamma(E+\omega) \Gamma_{\alpha}(E)|g_{\downarrow}^{r}\left(E\right)|^{2} |G_{\uparrow}^{0r} (E+\omega)|^{2} \big[f(E) - f(E+\omega) \big].
\label{eq:current7}
\end{equation}
\section{Check charge current}
In parametric pumping setup, left and right leads are identical, i.e. $f_{L}(E) = F_{R}(E)$, no charge current is induced. Notice that
\begin{equation}
\begin{split}
g_{\uparrow}^{r}(E) G_{\downarrow}^{0r}(E-\omega) &= \frac{ G_{\uparrow}^{0r}(E) G_{\downarrow}^{0r}(E-\omega)} {1-\gamma^{2} G_{\uparrow}^{0r}(E) G_{\downarrow}^{0r}(E-\omega)} \\
&= G_{\uparrow}^{0r}(E) g_{\downarrow}^{r}(E-\omega)
\end{split}
\end{equation}
then the spin-up current is
\begin{equation}
I_{\alpha\uparrow,\uparrow}&=-\frac{e}{\hbar}\int dE \gamma^{2}\Gamma(E-\omega) \Gamma_{\alpha}(E)|g_{\downarrow}^{r}\left(E-\omega\right)|^{2} |G_{\uparrow}^{0r} (E)|^{2} \big[f(E) - f(E-\omega) \big].
\end{equation}
The integral keeps the same when the independent variable is shifted, we make $E\rightarrow E+\omega$,
\begin{equation}
I_{\alpha\uparrow,\uparrow}&=-\frac{e}{\hbar}\int dE \gamma^{2}\Gamma(E) \Gamma_{\alpha}(E+\omega)|g_{\downarrow}^{r}\left(E\right)|^{2} |G_{\uparrow}^{0r} (E+\omega)|^{2} \big[f(E+\omega) - f(E) \big].
\end{equation}
Note that $\Gamma(E) = \Gamma_{L}(E) + \Gamma_{R}(E) = 2\Gamma_{\alpha}(E)$, so if $\Gamma_{\alpha}(E+\omega) = \Gamma_{\alpha}(E-\omega)$, we get
\begin{equation}
I_{\alpha\uparrow,\uparrow} + I_{\alpha\downarrow,\downarrow} = 0.
\end{equation}




\newpage


% \section{Appendix}
% \subsection{Double Fourier transformation}
% Note the double Fourier transformation, if
% \begin{equation}
% F(t)=\int d t_{1} G_{1}\left(t, t_{1}\right) G_{2}\left(t_{1}, t\right)
% \end{equation}
% then
% \begin{equation}
% F(\omega)=\int \frac{d E}{2 \pi} \frac{d E^{\prime}}{2 \pi} G_{1}\left(E+\omega, E^{\prime}\right) G_{2}\left(E^{\prime}, E\right).
% \end{equation}
% If
% \begin{equation}
% F\left(t_{1}, t_{2}\right)=\int d t G_{1}\left(t_{1}, t\right) G_{2}\left(t, t_{2}\right),
% \end{equation}
% then
% \begin{equation}
% F\left(E_{1}, E_{2}\right)=\int\frac{d E}{2\pi} G_{1}\left(E_{1}, E\right) G_{2}\left(E, E_{2}\right).
% \end{equation}
% If $F(t_{1}, t_{2})$ depends only on $t_{1}-t_{2}$,
% \begin{equation}
% F\left(E_{1}, E_{2}\right)=2\pi F\left(E_{1}\right)\delta(E_{1}-E_{2})
% \end{equation}
% Note that
% \begin{equation}
% \int_{-\infty}^{\infty} \delta(x) d x=1.
% \end{equation}
% \begin{equation}
% \int_{a-\epsilon}^{a+\epsilon} f(x) \delta(x-a) d x=f(a)
% \end{equation}
% \section{Adiabatic regime($\omega$ is small)}

% So, the charge current is given by (why? DC?)
% \begin{equation}
% d Q_{\alpha \sigma}(t) / d t=q \int \frac{d E}{2 \pi}\left(-\partial_{E} f\right)\left[\Gamma_{\alpha} \mathbf{G}^{r}(t) \mathbf{\Delta} \mathbf{G}^{a}(t)\right]_{\sigma \sigma}
% \end{equation}








\begin{thebibliography}{10}
\bibitem{ref1}
Y, K, Kato. Observation of the Spin Hall Effect in Semiconductors[J]. Science, 2004.
\bibitem{Jauho}
Antti-Pekka Jauho, Quantum Kinetics in Transport and Optics of Semiconductors, P188.
\bibitem{Baigeng}
PRB 67, 092408 (2003).
\end{thebibliography}

\end{CJK}
\end{document}
