\documentclass[11pt,a4paper]{article}
\usepackage{xcolor}
\usepackage{CJKutf8}
\usepackage{graphicx}
\usepackage{amsmath}
\usepackage{braket}
\usepackage{caption}
\usepackage{bm}
\usepackage{geometry}
\geometry{a4paper,scale=0.8}

\captionsetup{font={scriptsize}}

\begin{CJK}{UTF8}{gbsn}
\begin{document}

\title{Notes on PRB.67.092408}
% \author{Li Gaoyang}
\date{}
% \date{\today}
\maketitle
% \tableofcontents

% \newpage
\section{spin field effect transistor}
A type of quantum field effect transistor that operates purely on the flow of spin current in the absence of charge current. The rotating field induces a time-independent dc spin current, and at the same time generates no charge cur- rent. The physical principle of our SFET is due to a spin flip mecha- nism provided by the field.
\section{Hamiltonian}
A rotating magnetic field is
\begin{equation}
B_{x} = B_{0}\rm{sin}\theta ~\rm{cos}(\omega t)
\end{equation}
\begin{equation}
B_{y} = B_{0}\rm{sin}\theta~\rm{sin}(\omega t)
\end{equation}
\begin{equation}
B_{z} = B_{0}\rm{cos}\theta.
\end{equation}
The Hamiltonian of system is
\begin{equation}
\begin{split}
H=&\sum_{k, \sigma, \alpha=L, R} \epsilon_{k} C_{k \alpha \sigma}^{+} C_{k \alpha \sigma}+\sum_{\sigma}\left[\epsilon+\sigma B_{0} \cos \theta\right] d_{\sigma}^{+} d_{\sigma}\\
&+H^{\prime}(t)+\sum_{k, \sigma, \alpha=L, R}\left[T_{k \alpha} C_{k \alpha \sigma}^{+} d_{\sigma}+\mathrm{c.c.}\right]
\end{split}
\end{equation}
We assume that there is only one orbit in the scattering region.
\[
\epsilon_{Lk} = \epsilon_{Rk} = \epsilon_{k}.
\]
A counterclock-wise rotating field allows a spin-down electron to absorb a photon and flip to spin-up, and it does not allow a spin-up electron to absorb a photon and flip to spin-down.
\begin{equation}
H^{\prime}(t)=\gamma\left[\exp (-i \omega t) d_{\uparrow}^{+} d_{\downarrow}+\exp (i \omega t) d_{\downarrow}^{+} d_{\uparrow}\right]
\end{equation}
\begin{equation}
\gamma=B_{0} \sin \theta
\end{equation}
The scattering region is characterized by an energy level$\epsilon = \epsilon_{0} - qV_{g}$, controlled by the gate voltage $V_{g}$.

We solve the transport properties (charge and spin currents) of the model in both adiabatic and nonadiabatic regimes using the stan- dard Keldysh nonequilibrium Green’s function technique.
\section{Operator evolution}
\begin{equation}
\begin{split}
\frac{d}{dt}\hat{N}_{\alpha k\sigma} &= \frac{i}{\hbar}[H, C_{\alpha k\sigma}^{\dag}C_{\alpha k\sigma}] = \left[\sum_{k', \sigma', \alpha'=L, R}\left[T_{k' \alpha'} C_{k' \alpha' \sigma'}^{+} d_{\sigma'}+\mathrm{c.c.}\right], C_{\alpha k\sigma}^{\dag}C_{\alpha k\sigma}\right]\\
&=\frac{i}{\hbar}\sum_{k', \sigma', \alpha'=L, R}\left[ -T_{k' \alpha'} C_{k' \alpha' \sigma'}^{+} d_{\sigma'}\delta_{\alpha\alpha'}\delta_{kk'}\delta_{\sigma\sigma'}+\mathrm{c.c.}\right]\\
&=\frac{i}{\hbar}[-T_{k \alpha} C_{k \alpha \sigma}^{\dag} d_{\sigma} + T_{k \alpha}^{*} d_{\sigma}^{\dag}C_{k \alpha \sigma}]
\end{split}
\end{equation}
So, the charge current is given by
\begin{equation}
\begin{split}
I_{\alpha\sigma}(t) &= e\langle\frac{d}{dt}\hat{N}_{\alpha k\sigma}(t)\rangle \\
&=\frac{ie}{\hbar}(\langle -T_{k \alpha} C_{k \alpha \sigma}^{\dag}(t) d_{\sigma}(t)\rangle + \langle T_{k \alpha}^{*} d_{\sigma}^{\dag}(t)C_{k \alpha \sigma}(t)\rangle)
\end{split}
\end{equation}
\section{Adiabatic regime($\omega$ is small)}
Equation of motion of particle operator $\hat{N}_{\alpha k\sigma}$ in the lead $\alpha$ is
\begin{equation}
\begin{split}
\frac{d}{dt}\hat{N}_{\alpha k\sigma} &= \frac{i}{\hbar}[H, c_{\alpha k\sigma}^{\dag}c_{\alpha k\sigma}] = \left[\sum_{k', \sigma', \alpha'=L, R}\left[t_{k' \alpha'} c_{k' \alpha' \sigma'}^{\dag} d_{\sigma'}+\mathrm{c.c.}\right], c_{\alpha k\sigma}^{\dag}c_{\alpha k\sigma}\right]\\
&=\frac{i}{\hbar}\sum_{k', \sigma', \alpha'=L, R}\left[ -t_{k' \alpha'} c_{k' \alpha' \sigma'}^{\dag} d_{\sigma'}\delta_{\alpha\alpha'}\delta_{kk'}\delta_{\sigma\sigma'}+\mathrm{c.c.}\right]\\
&=\frac{i}{\hbar}[-t_{k \alpha} c_{k \alpha \sigma}^{\dag} d_{\sigma} + t_{k \alpha}^{*} d_{\sigma}^{\dag}c_{k \alpha \sigma}]
\end{split}
\end{equation}
So, the charge current is given by
\begin{equation}
\begin{split}
I_{\alpha\sigma}(t) &= e\langle\frac{d}{dt}\hat{N}_{\alpha k\sigma}(t)\rangle \\
&=\frac{ie}{\hbar}(\langle -t_{k \alpha} c_{k \alpha \sigma}^{\dag}(t) d_{\sigma}(t)\rangle + \langle t_{k \alpha}^{*} d_{\sigma}^{\dag}(t)c_{k \alpha \sigma}(t)\rangle)
\end{split}
\end{equation}
Define the lesser Green's function
\begin{equation}
G_{\sigma',k\alpha\sigma}^{<}(\tau,\tau') = i\langle C_{k\alpha\sigma}^{\dag}(\tau') d_{\sigma'}(\tau)\rangle .
\end{equation}
More generally, we define the contour Green's function
\begin{equation}
G_{\sigma',k\alpha\sigma}(\tau,\tau') = -i\langle  d_{\sigma'}(\tau) C_{k\alpha\sigma}^{\dag}(\tau')\rangle .
\end{equation}
Following Jauho's notation~\cite{Jauho}
\begin{equation}
G_{n,k\alpha}(\tau,\tau')=\sum_{m} \int d \tau_{1} G_{n m}\left(\tau, \tau_{1}\right) t_{k \alpha m}^{*} g_{k \alpha}\left(\tau_{1}, \tau^{\prime}\right)
\end{equation}
we have???
\begin{equation}
G_{n,k\alpha}(\tau,\tau')=\sum_{m} \int d \tau_{1} G_{n m}\left(\tau, \tau_{1}\right) t_{k \alpha m}^{*} g_{k \alpha}\left(\tau_{1}, \tau^{\prime}\right)
\end{equation}

So, the charge current is given by (why? DC?)
\begin{equation}
d Q_{\alpha \sigma}(t) / d t=q \int \frac{d E}{2 \pi}\left(-\partial_{E} f\right)\left[\Gamma_{\alpha} \mathbf{G}^{r}(t) \mathbf{\Delta} \mathbf{G}^{a}(t)\right]_{\sigma \sigma}
\end{equation}


\begin{thebibliography}{10}
\bibitem{ref1}
Y, K, Kato. Observation of the Spin Hall Effect in Semiconductors[J]. Science, 2004.
\bibitem{Jauho}
Antti-Pekka Jauho, Quantum Kinetics in Transport and Optics of Semiconductors, P188.
\end{thebibliography}

\end{CJK}
\end{document}
