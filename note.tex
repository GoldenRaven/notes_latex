\documentclass[11pt,a4paper]{article}
\usepackage{xcolor}
\usepackage{CJKutf8}
\usepackage{graphicx}
\usepackage{amsmath}
\usepackage{braket}
\usepackage{caption}
\usepackage{bm}
\usepackage{geometry}
\geometry{a4paper,scale=0.8}

\captionsetup{font={scriptsize}}

\begin{CJK}{UTF8}{gbsn}
\begin{document}

\title{Notes on PRB.67.092408}
% \author{Li Gaoyang}
\date{}
% \date{\today}
\maketitle
% \tableofcontents

% \newpage
\section{spin field effect transistor}
A type of quantum field effect transistor that operates purely on the flow of spin current in the absence of charge current. The rotating field induces a time-independent dc spin current, and at the same time generates no charge cur- rent. The physical principle of our SFET is due to a spin flip mecha- nism provided by the field.
\section{Hamiltonian}
A rotating magnetic field is
\begin{equation}
B_{x} = B_{0}\rm{sin}\theta ~\rm{cos}(\omega t)
\end{equation}
\begin{equation}
B_{y} = B_{0}\rm{sin}\theta~\rm{sin}(\omega t)
\end{equation}
\begin{equation}
B_{z} = B_{0}\rm{cos}\theta.
\end{equation}
The Hamiltonian of system is
\begin{equation}
\begin{split}
H=&\sum_{k, \sigma, \alpha=L, R} \epsilon_{k} C_{k \alpha \sigma}^{+} C_{k \alpha \sigma}+\sum_{n\sigma}\left[\epsilon_{n}+\sigma B_{0} \cos \theta\right] d_{n\sigma}^{+} d_{n\sigma}\\
&+H^{\prime}(t)+\sum_{k,n, \sigma, \alpha=L, R}\left[T_{k \alpha n} C_{k \alpha \sigma}^{+} d_{n\sigma}+\mathrm{c.c.}\right]
\end{split}
\end{equation}
We assume that there are multiple orbits in the scattering region, which is different from the original paper, in which only one orbit is considered. Energy level of lead $\epsilon_{Lk} = \epsilon_{Rk} = \epsilon_{k}$.

A counterclock-wise rotating field allows a spin-down electron to absorb a photon and flip to spin-up, and it does not allow a spin-up electron to absorb a photon and flip to spin-down.
\begin{equation}
H^{\prime}(t)=\sum_{n}\gamma\left[\exp (-i \omega t) d_{n\uparrow}^{+} d_{n\downarrow}+\exp (i \omega t) d_{n\downarrow}^{+} d_{n\uparrow}\right]
\end{equation}
\begin{equation}
\gamma=B_{0} \sin \theta
\end{equation}
The scattering region is characterized by an energy level $\epsilon_{n} = \epsilon_{n}^{0} - qV_{g}$, controlled by the gate voltage $V_{g}$.

We solve the transport properties (charge and spin currents) of the model in both adiabatic and nonadiabatic regimes using the standard Keldysh nonequilibrium Green’s function technique.
\section{Operator evolution}
EoM of $d_{\sigma}^{\dag}$ is
\begin{equation}
\partial_{t'} d_{n\sigma}^{\dag}(t') = i[H, d_{n\sigma}^{\dag}]
\label{eq:eom1}
\end{equation}
central dot part:
\begin{equation}
[\sum_{n'\sigma'}\left[\epsilon_{n'}+\sigma B_{0} \cos \theta\right] d_{n'\sigma'}^{\dag} d_{n'\sigma'}, d_{n\sigma}^{\dag}] =(\epsilon_{n}+\sigma B_{0} \cos \theta) d_{n\sigma}^{\dag}
\label{eq:part1}
\end{equation}
rotating magnetic field part:
\begin{equation}
\begin{split}
[d_{n'\uparrow}^{\dag} d_{n'\downarrow}, d_{n\sigma}^{\dag}] &= d_{n'\uparrow}^{\dag}\{d_{n'\downarrow}, d_{n\sigma}^{\dag}\} - \{d_{n'\uparrow}^{\dag}, d_{n\sigma}^{\dag}\}d_{n'\downarrow} \\
&=d_{\uparrow}^{\dag}\delta_{nn'}\delta_{\sigma\downarrow},
\end{split}
\end{equation}
\begin{equation}
\begin{split}
[d_{n'\downarrow}^{\dag} d_{n'\uparrow}, d_{n\sigma}^{\dag}] &= d_{n'\downarrow}^{\dag}\{d_{n'\uparrow}, d_{n\sigma}^{\dag}\} - \{d_{n'\downarrow}^{\dag}, d_{n\sigma}^{\dag}\}d_{n'\uparrow} \\
&=d_{n'\uparrow}^{\dag}\delta_{nn'}\delta_{\sigma\uparrow}.
\end{split}
\end{equation}
Then
\begin{equation}
[H^{\prime}(t), d_{n\sigma}^{\dag}] = \gamma(e^{-i \omega t} d_{n\uparrow}^{\dag} \delta_{\sigma\downarrow} + e^{i \omega t} d_{n\downarrow}^{\dag} \delta_{\sigma\uparrow})
\end{equation}
Coupling part
\begin{equation}
\sum_{k, n',\sigma', \alpha=L, R}\left[T_{n'k \alpha} C_{k \alpha \sigma'}^{\dag} d_{n'\sigma'}+\mathrm{c.c.}\right, d_{n\sigma}^{\dag}] = \sum_{k, \alpha=L, R}T_{k \alpha n} C_{k \alpha \sigma}^{\dag}
\end{equation}
Substitute these three parts into Eq. (\ref{eq:eom1}), we get
\begin{equation}
\partial_{t'} d_{n\sigma}^{\dag}(t') = i[(\epsilon_{n}^{0}+\sigma B_{0} \cos \theta) d_{n\sigma}^{\dag} + \gamma(e^{-i \omega t} d_{n\uparrow}^{\dag} \delta_{\sigma\downarrow} + e^{i \omega t} d_{n\downarrow}^{\dag} \delta_{\sigma\uparrow}) + \sum_{k, \alpha=L, R}T_{k \alpha n} C_{k \alpha \sigma}^{\dag}]
\end{equation}
Equation of motion of particle operator $\hat{N}_{\alpha k\sigma}$ in the lead $\alpha$ is
\begin{equation}
\begin{split}
\frac{d}{dt}\hat{N}_{\alpha k\sigma} &= \frac{i}{\hbar}[H, C_{\alpha k\sigma}^{\dag}C_{\alpha k\sigma}] = \left[\sum_{k', \sigma', \alpha'=L, R}\left[T_{k' \alpha'} C_{k' \alpha' \sigma'}^{+} d_{\sigma'}+\mathrm{c.c.}\right], C_{\alpha k\sigma}^{\dag}C_{\alpha k\sigma}\right]\\
&=\frac{i}{\hbar}\sum_{k', \sigma', \alpha'=L, R}\left[ -T_{k' \alpha'} C_{k' \alpha' \sigma'}^{+} d_{\sigma'}\delta_{\alpha\alpha'}\delta_{kk'}\delta_{\sigma\sigma'}+\mathrm{c.c.}\right]\\
&=\frac{i}{\hbar}[-T_{k \alpha} C_{k \alpha \sigma}^{\dag} d_{\sigma} + T_{k \alpha}^{*} d_{\sigma}^{\dag}C_{k \alpha \sigma}]
\end{split}
\end{equation}
\section{Charge current }
So, the charge current due to L(R) lead with spin $\sigma$ is given by
\begin{equation}
\begin{split}
I_{L\sigma}(t) &= e\langle\frac{d}{dt}\hat{N}_{\sigma}(t)\rangle \\
&=\frac{ie}{\hbar}\sum_{kn\alpha\in L}(\langle -T_{k \alpha n} C_{k \alpha \sigma}^{\dag}(t) d_{n\sigma}(t)\rangle + \langle T_{k \alpha n}^{*} d_{n\sigma}^{\dag}(t)C_{k \alpha \sigma}(t)\rangle)
\end{split}
\end{equation}
Define the lesser Green's function
\begin{equation}
G_{n\sigma',k\alpha\sigma}^{<}(\tau,\tau') = i\langle C_{k\alpha\sigma}^{\dag}(\tau') d_{n\sigma'}(\tau)\rangle ,
\label{eq:contour}
\end{equation}
the current becomes
\begin{equation}
I_{L\sigma}(t)=\frac{-e}{\hbar}\sum_{kn\alpha\in L}(T_{k\alpha n}G_{n,k\alpha\sigma}^{<}(t,t) - T_{k \alpha n}^{*} G_{k\alpha,n}^{<}(t,t)\rangle)
\end{equation}
More generally, we define the contour Green's function
\begin{equation}
G_{n\sigma',k\alpha\sigma}(\tau,\tau') = -i\langle  d_{n\sigma'}(\tau) C_{k\alpha\sigma}^{\dag}(\tau')\rangle .
\end{equation}
EoM of operator $C_{k\alpha \sigma}^{\dag}$ is
\begin{equation}
\partial_{t'} C_{k\alpha \sigma}^{\dag}(t') = i[H, C_{k\alpha \sigma}^{\dag}] = i( \varepsilon_{k}C_{k\alpha \sigma}^{\dag} + \sum_{n}T_{k\alpha n}d_{n\sigma}^{\dag})
\end{equation}
The equation-of-motion for the time-ordered Green function
\begin{equation}
\begin{split}
-i \frac{\partial}{\partial t^{\prime}} G_{n\sigma',k\alpha\sigma}^{t} \left(t,t^{\prime}\right)&= \delta(t-t')\langle \{d_{n\sigma'}, C_{k\alpha\sigma}^{\dag}\}\rangle - \langle T_{c}d_{n\sigma'}\partial_{t'}C_{k\alpha\sigma}^{\dag}\rangle \\
&=\varepsilon_{k} G_{n\sigma',k\alpha\sigma}^{t}\left(t,t^{\prime}\right)+ \sum_{m} T_{k\alpha m}^{*} G_{n\sigma',m\sigma}^{t}(t,t^{\prime})
\end{split}
\end{equation}
So, we have
\begin{equation}
(-i \frac{\partial}{\partial t^{\prime}} -\varepsilon_{k})G_{n\sigma',k\alpha\sigma}^{t} \left(t,t^{\prime}\right)= \sum_{m} T_{k\alpha m}^{*} G_{n\sigma',m\sigma}^{t}(t,t^{\prime})
\label{eq:free}
\end{equation}
in which
\begin{equation}
G_{n\sigma',m\sigma}^{t}(t,t^{\prime}) = -i\langle T_{c}d_{n\sigma'}(t)d_{m\sigma}^{\dag}(t')\rangle.
\end{equation}
Similarly, we evaluate the EoM for free Green's function $g_{k\alpha\sigma}^{t}(t,t')$ in lead $\alpha$ (note that $H=\sum_{k\sigma\alpha} \epsilon_{k} C_{k \alpha \sigma}^{\dag} C_{k \alpha \sigma}$).
\begin{equation}
\begin{split}
-i \frac{\partial}{\partial t^{\prime}} g_{k\alpha\sigma}^{t} \left(t,t^{\prime}\right)&= \delta(t-t')\langle \{C_{k\alpha\sigma}, C_{k\alpha\sigma}^{\dag}\}\rangle - \langle T_{c}C_{k\alpha\sigma}\partial_{t'}C_{k\alpha\sigma}^{\dag}\rangle \\
&=\delta(t-t') + \varepsilon_{k} g_{k\alpha\sigma}^{t}\left(t,t^{\prime}\right),
\end{split}
\end{equation}
we have
\begin{equation}
(-i \frac{\partial}{\partial t^{\prime}}-\varepsilon_{k}) g_{k\alpha\sigma}^{t} \left(t,t^{\prime}\right)=\delta(t-t').
\label{eq:free-delta}
\end{equation}
Substitute Eq. (\ref{eq:free-delta}) into Eq. (\ref{eq:free}) and integrate on both sides, we get an equation analogous to Jauho's notation~\cite{Jauho},
\begin{equation}
G_{n,k\alpha}(\tau,\tau')=\sum_{m} \int d \tau_{1} G_{n m}\left(\tau, \tau_{1}\right) t_{k \alpha m}^{*} g_{k \alpha}\left(\tau_{1}, \tau^{\prime}\right),
\nonumber
\end{equation}
we have
\begin{equation}
G_{n\sigma',k\alpha\sigma}^{t}(t,t')=\sum_{m}\int d t_{1} G_{n\sigma',m\sigma}\left(t, t_{1}\right) T_{k \alpha m}^{*} g_{k \alpha\sigma}^{t}\left(\tau_{1}, \tau^{\prime}\right).
\end{equation}
When there is only one orbit presents, this equation reduces to
\begin{equation}
G_{\sigma',k\alpha\sigma}^{t}(t,t')=\int d t_{1} G_{\sigma',\sigma}\left(t, t_{1}\right) T_{k \alpha}^{*} g_{k \alpha\sigma}^{t}\left(\tau_{1}, \tau^{\prime}\right).
\end{equation}
Since the contour Green's function has the same structure as real-time Green's function, the we have relation
\begin{equation}
G_{n\sigma',k\alpha\sigma}(\tau,\tau')=\sum_{m}\int d \tau_{1} G_{n\sigma',m\sigma}\left(\tau, \tau_{1}\right) T_{k \alpha m}^{*} g_{k \alpha\sigma}\left(\tau_{1}, \tau^{\prime}\right)
\end{equation}
where $G_{n\sigma',k\alpha\sigma}(\tau,\tau')$ is contour Green's function defined in Eq. (\ref{eq:contour}), and similarly the contour Green's function for non-interacting lead is defined as
\begin{equation}
g_{k\alpha\sigma}(\tau,\tau') = -i\langle T_{c}C_{k\alpha\sigma}(\tau)C_{k\alpha\sigma}^{\dag}(\tau')\rangle
\end{equation}
After analytic continuation, the current is formulated as
\begin{equation}
\begin{aligned}
I_{\alpha\sigma}(t) &=-\frac{e}{\hbar} \int d t_{1} \operatorname{Tr}\left[G^{r}\left(t, t_{1}\right) \Sigma_{\alpha}^{<}\left(t_{1}, t\right)\right.\\
&\left.+G^{<}\left(t, t_{1}\right) \Sigma_{\alpha}^{a}\left(t_{1}, t\right)\right]+h . c .
\end{aligned}
\end{equation}
Dyson equation:
\begin{equation}
G = g + g\Sigma G
\end{equation}
Note the double Fourier transformation, if
\begin{equation}
F(t)=\int d t_{1} G_{1}\left(t, t_{1}\right) G_{2}\left(t_{1}, t\right)
\end{equation}
then
\begin{equation}
F(\omega)=\int \frac{d E}{2 \pi} \frac{d E^{\prime}}{2 \pi} G_{1}\left(E+\omega, E^{\prime}\right) G_{2}\left(E^{\prime}, E\right).
\end{equation}
If
\begin{equation}
F\left(t_{1}, t_{2}\right)=\int d t G_{1}\left(t_{1}, t\right) G_{2}\left(t, t_{2}\right),
\end{equation}
then
\begin{equation}
F\left(E_{1}, E_{2}\right)=\int\frac{d E}{2\pi} G_{1}\left(E_{1}, E\right) G_{2}\left(E, E_{2}\right).
\end{equation}
Following Eq. (224) in WangJian's note, its Fourier transformation is
\begin{equation}
\begin{split}
I_{\alpha\sigma}(\omega)=-\frac{e}{\hbar}\int \frac{d E}{2 \pi} \frac{d E^{\prime}}{2 \pi} \rm{Tr}\left[&G^{r}\left(E+\omega, E^{\prime}\right) \Sigma_{\alpha}^{<}\left(E^{\prime}, E\right) \\
&+ G^{<}(E+\omega, E')\Sigma_{\alpha}^{a}(E', E) ] + c.c.
\end{split}
\end{equation}
Here, $G^{r,<} \equiv G_{n\sigma',m\sigma}^{r,<}$, and matrix element
\begin{equation}
\Sigma_{\alpha,mn}^{\gamma}(t_{1}, t_{2}) = \sum_{k} T_{k\alpha m}^{*}(t_{1}) g_{k\alpha}^{\gamma}(t_{1}, t_{2}) T_{k\alpha n}(t_{2}),
\end{equation}
in which, $g_{k\sigma}$ is the free propagator of lead. From Dyson equation, we have (different from Eq. (77) in Chap. II ???)
\begin{equation}
G^{<}=G^{r} \Sigma^{<} G^{a}
\end{equation}
whose Fourier transformation gives
\begin{equation}
G^{<}(E_{1},E_{2})=\int\frac{dE}{2\pi}\frac{dE'}{2\pi}G^{r}(E_{1}, E) \Sigma^{<}(E, E') G^{a}(E', E_{2})
\end{equation}
For $G^{r}$, the analytic continuation gives
\begin{equation}
G^{r}=g^{r}+g^{r} H' G^{r}
\end{equation}
\begin{equation}
G^{r}(E_{1}, E_{2})=g^{r}(E_{1}, E_{2})+\int\frac{dE}{2\pi}\frac{dE'}{2\pi}g^{r}(E_{1}, E) H^{'}(E, E') G^{r}(E', E_{2})
\label{eq:Gr}
\end{equation}
\begin{equation}
g^{r}\left(E_{1}, E_{2}\right) = 2\pi g^{r}\left(E_{1}\right) \delta(E_{1}-E_{2})
\end{equation}
in which, $g^{r}$ is the free propagator of central dot(?). Using Eq. (\ref{eq:part1}), we have
\begin{equation}
\begin{split}
g_{n\sigma}^{r}(t,t') =& -i\theta(t-t')\langle \{d_{n\sigma}(t),d_{n\sigma}^{\dag}(t')\} \rangle_{0} \\
=&-i\theta(t-t') e^{-i(\epsilon_{n}+\sigma B_{0} \cos \theta) (t-t')}
\end{split}
\end{equation}
and
\begin{equation}
g_{n\sigma}^{r}(E) = \frac{1}{E-(\epsilon_{n}+\sigma B_{0} \cos \theta)+i 0^{+}}
\end{equation}
In the spin space \{$d_{\uparrow}^{\dag}$, $d_{\downarrow}^{\dag}$; $d_{\uparrow}$, $d_{\uparrow}$\},
\begin{equation}
G^{r}\equiv\left(\begin{array}{cc}
G_{n\uparrow,m\uparrow}^{r} & G_{n\uparrow,m\downarrow}^{r} \\
G_{n\downarrow,m\uparrow}^{r} & G_{n\downarrow,m\downarrow}^{r}
\end{array}\right)
\end{equation}
\begin{equation}
g^{r}(E)=\left(\begin{array}{cc}
g_{n\uparrow}^{r}(E) &0 \\
0 & g_{n\downarrow}^{r}(E)
\end{array}\right).
\end{equation}
$H'$ is given by
\begin{equation}
H^{\prime}=\left(\begin{array}{cc}
0 & \gamma e^{-i \omega t} \\
\gamma e^{i \omega t} & 0
\end{array}\right)
\end{equation}
we have
\begin{equation}
H^{\prime}(E, E')=\left(\begin{array}{cc}
0 & 2\pi\gamma \delta(E-\omega) \\
2\pi\gamma \delta(E+\omega) & 0
\end{array}\right) \delta(E-E').
\end{equation}
Substitute these equations into Eq. (\ref{eq:Gr}), we get
\begin{equation}
\begin{split}
\left(\begin{array}{cc}
G_{n\uparrow,m\uparrow}^{r} & G_{n\uparrow,m\downarrow}^{r} \\
G_{n\downarrow,m\uparrow}^{r} & G_{n\downarrow,m\downarrow}^{r}
\end{array}\right) = &
\left(\begin{array}{cc}
g_{n\uparrow}^{r}(E_{1}) &0 \\
0 & g_{n\downarrow}^{r}(E_{1})
\end{array}\right) 2\pi\delta(E_{1}-E_{2})\\
&+ \int\frac{dE}{2\pi}\frac{dE'}{2\pi}\left(\begin{array}{cc}
g_{n\uparrow}^{r}(E_{1}) &0 \\
0 & g_{n\downarrow}^{r}(E_{1})
\end{array}\right) 2\pi\delta(E_{1}-E)\delta(E-E')\\
&\times
\left(\begin{array}{cc}
0 & 2\pi\gamma \delta(E-\omega) \\
2\pi\gamma \delta(E+\omega) & 0
\end{array}\right)
 \left(\begin{array}{cc}
G_{n\uparrow,m\uparrow}^{r} & G_{n\uparrow,m\downarrow}^{r} \\
G_{n\downarrow,m\uparrow}^{r} & G_{n\downarrow,m\downarrow}^{r}
\end{array}\right)
\end{split}
\end{equation}






\section{Adiabatic regime($\omega$ is small)}

So, the charge current is given by (why? DC?)
\begin{equation}
d Q_{\alpha \sigma}(t) / d t=q \int \frac{d E}{2 \pi}\left(-\partial_{E} f\right)\left[\Gamma_{\alpha} \mathbf{G}^{r}(t) \mathbf{\Delta} \mathbf{G}^{a}(t)\right]_{\sigma \sigma}
\end{equation}


\begin{thebibliography}{10}
\bibitem{ref1}
Y, K, Kato. Observation of the Spin Hall Effect in Semiconductors[J]. Science, 2004.
\bibitem{Jauho}
Antti-Pekka Jauho, Quantum Kinetics in Transport and Optics of Semiconductors, P188.
\end{thebibliography}

\end{CJK}
\end{document}
